\cvsection{Expérience}
\vspace{-1em}

\begin{minipage}[t]{.49\linewidth}\vspace{-1pt}
    \cvsubsection{\color{awesome-concrete}Académique}\vspace{-2pt}
% 
    \begin{cventries}
% 
        \cventryvar
            {Thèse de Doctorat en Chimie}
            {Université Côte d'Azur, France}
            {Institut de Chimie de Nice}
            {2018 -- 2021}
            {
            \begin{cvitems}
                \item{Bases Moléculaires de la Perception Chimiosensorielle}
                \item{Dirigée par Dr. Sébastien \textsc{Fiorucci} \& Pr. Serge \textsc{Antonczak}}
                \item{Compétences: Chémoinformatique, Modélisation Moléculaire, Machine-Learning, Programmation, RCPG chémosensoriel}
            \end{cvitems}
            }
%
        \cventryvar
            {Stage}
            {Université Côte d'Azur, France}
            {Institut de Chimie de Nice}
            {Février -- Juillet 2018}
            {
            \begin{cvitems}
                \item{Développement de modèles QSAR pour la prédiction de propriétés olfactives et gustatives de molécules chez l'homme et l'insecte}
                \item{Encadré par Dr. S. \textsc{Fiorucci}}
            \end{cvitems}
            }
%
        \cventryvar
            {Stage}
            {Université de Strasbourg, France}
            {Institut de Science et d'Ingénierie Supramoléculaires}
            {Mars -- Juin 2017}
            {
            \begin{cvitems}
                \item{Automatisation de calculs d'énergie libre de liaison}
                \item{Application à un moteur moléculaire, la myosine}
                \item{Encadré par Dr. M. \textsc{Cecchini}}
            \end{cvitems}
            }
% 
        \cventryvar
            {Stage}
            {Université Côte d'Azur, France}
            {Institut de Chimie de Nice}
            {Juin 2016}
            {
            \begin{cvitems}
                \item{Modélisation par homologie des récepteurs au goût amer}
                \item{Encadré par Dr. S. \textsc{Fiorucci}}
            \end{cvitems}
            }
% 
    \end{cventries}
% 
\end{minipage}\nobreak
%
\begin{minipage}[t]{.02\linewidth}\vspace{0pt}
     \ 
\end{minipage}\nobreak
% 
\begin{minipage}[t]{.49\linewidth}\vspace{0pt}
    \cvsubsection{\color{awesome-concrete}Professionnelle}
% 
    \begin{cventries}
% 
        \cventryvar
            {Participant Google Summer of Code}
            {Télétravail}
            {MDAnalysis}
            {Juin -- Août 2020}
            {
            \vspace{-1.2\baselineskip}
            Développeur en charge du projet d'interopérabilité entre les modules MDAnalysis et RDKit: \vspace{1.2\baselineskip}
            \begin{cvitems}
                \item{Conversion entre les objets Python de MDAnalysis et RDKit}
                \item{Exploitation des fonctionnalités RDKit (requêtes SMARTS, calcul de descripteurs moléculaires... etc.) directement depuis MDAnalysis.}
            \end{cvitems}
            \vspace{1em}
            }
% 
        \cventryvar
            {Technicien}
            {Strasbourg, France}
            {Institut de Science et d'Ingénierie Supramoléculaires}
            {Juillet -- Août 2017}
            {
            \begin{cvitems}
                \item{Développement de workflows de chimie computationnelle pour le criblage virtuel.}
            \end{cvitems}
            \vspace{1em}
            }
% 
        \cventryvar
            {Auxiliaire d'été}
            {Monaco}
            {Banques}
            {2012 -- 2016}
            {
            \begin{cvitems}
                \item{Guichetier, opérateur de saisie}
                \item{Compagnie Monégasque de Banque, BNP Paribas Wealth \mbox{Management}, LCL Banque et Assurance}
            \end{cvitems}
            \vspace{1em}
            }
% 
    \end{cventries}
% 
\end{minipage}%